%% Erl�uterungen zu den Befehlen erfolgen unter
%% diesem Beispiel.
\documentclass{scrartcl}
 
\usepackage[utf8]{inputenc}
\usepackage[T1]{fontenc}
\usepackage[ngerman]{babel}
\usepackage[babel,german=quotes]{csquotes}
\usepackage{lmodern}
\usepackage{amsmath}
\usepackage{url}
\usepackage{lipsum}

\newcommand{\appname}{OASIS~}
\newcommand{\tomcathome}{\texttt{TOMCAT\_HOME}}
 
\title{Installationsanleitung \appname{}}
\subtitle{F\"ur Linux/Mac OS}

\begin{document}
 
\maketitle
\tableofcontents

%%%%%%%%%%%%%%%%%%%%%%%%%%%%%%%%%%%%%%%%%%%%%%%%%
\section{Versionierung}

\begin{center}
\begin{tabular}{ p{50pt} | p{50pt} | p{310pt} }
	\hline
	\textbf{Version} & \textbf{Datum} & \textbf{Grund f\"ur \"Anderungen}  \\
	\hline
	0.2	& 31.10.2013	& Anpassungen f\"ur die Systemanforderungen \\
  	0.1 	& 28.10.2013	& Initiale Version der Installationsanleitung 		 \\
	\hline
\end{tabular}
\end{center}

%%%%%%%%%%%%%%%%%%%%%%%%%%%%%%%%%%%%%%%%%%%%%%%%%
\section{Systemanforderungen}
F\"ur die Installation der Web-Applikation \appname sind folgende Systemanforderungen zu erf\"ullen.

\begin{itemize}
	\item Java 1.7
	\item Apache Tomcat 7.0
	\item Mindestens 1GB freier Arbeitsspeicher
	\item MySQL Server 5.5
\end{itemize}

Sie k\"onnen versuchen die Applikation auf neueren Versionen der angegebenen Software zu installieren, jedoch empfehlen wir Ihnen sich an die vorgegebenen Versionsnummern zu halten, da mit diesen das System getestet wurde.
 
%%%%%%%%%%%%%%%%%%%%%%%%%%%%%%%%%%%%%%%%%%%%%%%%%
\section{Installationsanleitung}
Folgende Schritte beschreiben, wie die Web-Application \appname auf Ihrem System installiert werden kann.

%%%%%%%%%%%%%%%%%%%%%%
\subsection{Installation Java}
Die Installation von Java ist abh\"angig von der jeweiligen Distribution Ihres Betriebssystems.
Es ist ausreichend eine Java Runtime Environment (JRE) zu installieren. 
Ein Java Development Kit (JDK) ist f\"ur die Ausf\"urung der Applikation nicht erforderlich.

Wir empfehlen f\"ur die Installation der JRE das Paketmanagementsystem Ihrer Linux-Distribution zu verwenden.

%%%%%%%%%%%%%%%%%%%%%%
\subsection{Installation MySQL}
F\"ur die Installation eines MySQL-Servers greifen Sie bitte ebenfalls auf das Paketverwaltungssystem Ihrer Linux-Distribution zur\"uck.
Nach der Installation legen Sie bitte ein neues Datenbankschema an (bspw. \texttt{idss}).
Dies erfolgt \"uber die MySQL-Query \texttt{CREATE DATABASE idss;}.
Legen Sie anschlie{\ss}end einen neuen Benutzer an, welcher alle Rechte auf der neu erstellten Datenbank bekommt:
\texttt{GRANT ALL PRIVILEGES ON idss.* TO 'idss'@'\%' IDENTIFIED BY 'idss';} \\

\textit{Erl\"auterung:} Es wird eine neue Datenbank mit dem Namen \texttt{idss} angelegt. 
Anschlie{\ss}end werden dem Benutzer \texttt{idss} auf alle Tbellen der Datenbank \texttt{idss} (gekenn-zeichnet durch folgenden Ausdruck: \texttt{idss.*}) lesende und schreibende Rechte vergeben. 
Der Ausdruck \texttt{IDENTIFIED BY 'idss'} legt das Passwort des Benutzers \texttt{idss} fest.

%%%%%%%%%%%%%%%%%%%%%%
\subsection{Installation Tomcat}
Der Applikationsserver Apache Tomcat kann von der Webseite \url{http://tomcat.apache.org} bezogen werden.
Empfohlen wird die \texttt{Version 7.0.47}, da mit dieser Version die Applikation getestet wurde.

\minisec{1. Schritt}
Erstellen Sie einen Ordner (bspw. \texttt{/opt/tomcat/}) in welchen Sie das heruntergeladene Archiv entpacken.
Der Ordner, in welches Sie das Archiv entpackt haben, wird im folgenden als \tomcathome~bezeichnet.
 
\minisec{2. Schritt}
\"Offnen Sie die Datei \texttt{server.xml} im Ordner \texttt{conf} und suchen Sie die Zeile welche folgenden Text enth\"alt: \texttt{<Connector port=\enquote{8080} \ldots~/>}.
F\"ugen Sie dort folgendes ein: \texttt{URIEncoding=\enquote{UTF-8}}. \\

\textit{Erl\"auterung:} Diese Einstellung wandelt die in einer URL angegebenen Eingaben in UTF-8-Encoding um und stellt sicher, dass der \"ubergebene Wert mit der richtigen Encodierung \"ubertragen wird.

\minisec{3. Schritt}
\"Offnen Sie die Datei \texttt{setenv.sh} im Ordner \texttt{bin}.
Sollte diese Datei nicht existieren, legen Sie diese an.
F\"ugen Sie nun folge Zeile in diese Datei ein:\\
\texttt{JAVA\_OPTS=\enquote{-Djavax.servlet.request.encoding=UTF-8 -Dfile.encoding=UTF-8 \\ -Xmx1024m -Xms1024m}} \\

\textit{Erl\"auterung:} Der Source-Code der Applikation wurde mit UTF-8-Encoding entwickelt und muss von der JVM als UTF-8-Encodierter Code ausgef\"uhrt werden. 
Da HTML-Seiten Umlaute enthalten k\"onnen, muss die JVM diese mit UTF-8 encodieren und an den Browser des Benutzers \"ubertragen, damit dieser keine fehlerhafte Ausgabe erh\"alt.
Die Parameter \texttt{Xmx} und \texttt{Xms} legen fest, wie viel Arbeitsspeicher die JVM allokieren darf. \texttt{Xms} legt fest, wie viel Arbeitsspeicher initial belegt werden und \texttt{Xmx} wie viel Arbeitsspeicher die JVM maximal belegen kann.

\minisec{4. Schritt}
Es bietet sich an die Installierte Tomcat-Instanz als Dienst (Service) einzurichten, sodass dieser bei einem Neustart automatisch hochgefahren wird.
Da die Installation als Dienst abh\"angig von Ihrer Plattform ist, bieten wir Ihnen ein Startup-Script an, welches unter Ubuntu 12.04 getestet wurde.

%%%%%%%%%%%%%%%%%%%%%%
\subsection{Installation Web-Applikation}
Im folgenden Abschnitt werden weiter \"Anderungen an Ihrer Tomcat-Installation notwendig.
Ersetzen Sie im folgenden die Variable \tomcathome~mit dem Pfad zu Ihrer Tomcat-Installation.

\minisec{1. Schritt}
Kopieren Sie den Kontext-Descriptor (\texttt{oasis.xml}) in den Ordner \\
\texttt{\tomcathome/conf/Catalina/localhost/}

\minisec{2. Schritt}
Legen Sie einen neuen Ordner im \tomcathome-Verzeichnis an (bspw. \texttt{mywebapps}) und kopieren sie die WAR-Datei in dieses Verzeichnis.

\minisec{3. Schritt}
\"Offnen Sie erneut den Kontext-Descriptor (s. Schritt 1) und passen Sie den Wert \texttt{docBase} so an, dass dieser den absoluten Pfad zur WAR-Datei aus Schritt 2 enth\"alt.

M\"ochten Sie, dass die Web-Applikation unter einem bestimmten Pfad erreichbar ist (bspw. \texttt{http://localhost:8080/app}), passen Sie den Wert \texttt{path} evenfalls an (bspw. \texttt{path=\enquote{/app}})

%%%%%%%%%%%%%%%%%%%%%%%%%%%%%%%%%%%%%%%%%%%%%%%%%
\section{Inbetriebnahme}
Nach abgeschlossener Installation starten Sie den Tomcat-Server \"uber das Script \texttt{catalina.sh} im \texttt{bin}-Verzeichnis: \texttt{catalina.sh run}







\end{document}